\documentclass[a4paper,12pt]{article}
\usepackage[framed,numbered,autolinebreaks,useliterate]{mcode}
\usepackage{CJKutf8}
\setlength{\headheight}{15pt} 
\usepackage{textcomp}
\usepackage{amsmath}
\usepackage{amssymb}
\usepackage{listings}
\usepackage{float}
\usepackage{xcolor}
\usepackage{color}
\usepackage{fancyhdr}
\usepackage{lastpage}
\usepackage{times}
\usepackage{mathptmx}
\usepackage{geometry}
\usepackage{booktabs}
\usepackage{graphicx}
\geometry{left=3.17cm,right=3.17cm,top=2.54cm,bottom=2.54cm}
\usepackage{indentfirst}
\setlength{\parindent}{2em}
\rhead{Page \thepage{} of \pageref{LastPage}}



\begin{document}
\begin{CJK*}{UTF8}{gbsn}



\section{实验课题}
下面利用蒙特卡罗方法计算一个冰淇淋的体积和质量.\par
假设冰淇淋的下部为一锥体而上部为一个半球, 锥面方程为$z^2=x^2+y^2$, 球面方程为$x^2+y^2+(z-1)^2=1$. 完成以下实验:\par 
\begin{enumerate}
\item 画出冰淇淋的图形.
\item 分别用确定性方法和蒙特卡罗方法计算冰淇淋的体积, 比较计算结果.
\item 假设冰淇淋的密度函数为为$f(x,y,z)=z\exp{[-(x^2+y^2+z^2)]}$, 分别用确定性方法和蒙特卡罗方法计算冰淇淋的质量, 比较计算结果.
\end{enumerate}



\section{图形绘制}

\subsection{理论分析}

\subsection{实验过程}




\section{体积计算}

\subsection{理论分析}

\subsection{实验过程}




\section{质量计算}


\subsection{理论分析}

\subsection{实验过程}


\section{附录}
\noindent\textbf{Lab03.m} 源代码
\vspace{-10pt}
\lstset{basicstyle=\ttfamily\footnotesize,escapechar=`}
\begin{lstlisting}
%Lab03.m
clear;clc;format long;
syms theta r phi;
countAll = 1000000;
countInIceCream = 0;
S = 0;
f = @(x,y,z) z*exp(-(x^2+y^2+z^2));

%draw the figure
z1 = @(x,y) sqrt(x.^2+y.^2);
figure('name','ice-cream figure');
g1 = ezmesh(z1,[-1/sqrt(2),1/sqrt(2),-1/sqrt(2),1/sqrt(2)],'circ');
hold on;
z2 = @(x,y) sqrt(1-x.^2-y.^2)+1;
g2 = ezmesh(z2,[-1/sqrt(2),1/sqrt(2),-1/sqrt(2),1/sqrt(2)],'circ');
title('ice-cream');
axis equal;

%generate random points
for i = 1:countAll
   w = [2*rand(1)-1,2*rand(1)-1,2*rand(1)];
   if (w(3)^2>w(1)^2+w(2)^2 && w(1)^2+w(2)^2+(w(3)-1)^2<1)
       countInIceCream = countInIceCream + 1;
       S = S + f(w(1),w(2),w(3));
   end
end

%calculate volume
V1 = eval(int(int(int(r*r*sin(theta),r,0,2*cos(theta)),...
    theta,0,pi/4),phi,0,2*pi));
V2 = 8 * countInIceCream / countAll;
fprintf('The actual volume of the ice-cream is %.6f.\n\n',V1);
fprintf('The estimated volume of the ice-cream is %.6f.\n\n',V2);


%calculate mass
M1 = eval(int(int(int(r^3*sin(theta)*cos(theta)*exp(-r^2),...
    r,0,2*cos(theta)),theta,0,pi/4),phi,0,2*pi));
M2 = S / countInIceCream * V2;
fprintf('The actual mass of the ice-cream is %.6f.\n\n',M1);
fprintf('The estimated mass of the ice-cream is %.6f.\n\n',M2);
\end{lstlisting}




\end{CJK*}
\end{document}
